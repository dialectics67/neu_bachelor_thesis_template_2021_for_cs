\chapter{算法的发展与原理}

\section{显著性对象检测的发展历史}

ABBA\cite{730558}
\section{另一个section}

先将图像分割 \cite{Felzenszwalb:2004:EGI:981793.981796}

\subsection{色彩直方图}

一些数学公式的例子\\

我们用两个颜色\inlinemath{c_k, c_i}之间的距离\inlinemath{D(c_k, c_i)}来定义他们的差异,则某个像素点的显著性值\inlinemath{S(P_k)}定义为
\insertmath{S(P_k) =  D(c_k, c_1) +  D(c_k, c_2) + \cdots +  D(c_k, c_N)\label{equ:saliency}}
这里是数学公式句子内部插入与句子间插入的两个例子


这里可以指向上面的算式\myref{equ:saliency}


\subsection{颜色量化}

文字

\subsection{另一个subsection}

\insertmath{D_R(R_m, R_n) = \sum_{i=m}^{n_1}\sum_{j=n}^{n_2}f(c_{m,i})f(c_{n,j})D(c_{m,i},c_{n,j})}
其中,\inlinemath{f(c_{m,i})}是第\inlinemath{i}个颜色\inlinemath{c_{m,i}}在第\inlinemath{k}个区域


\subsection{区域对比度}

